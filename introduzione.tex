\section{Introduzione}

\subsection{Statement}
\textbf{SplitManager} è un sistema software per la gestione delle spese condivise all'interno di gruppi di persone. L'applicazione consente a gruppi di amici, colleghi, parenti di tenere in modo organizzato il conto delle spese comuni e di semplificare i rimborsi.

Ruoli e funzionalità principali:
\begin{itemize}
    \item \textbf{User} (Utente Registrato): può registrarsi al sistema, effettuare il login, creare nuovi gruppi o unirsi a gruppi esistenti.
    \item \textbf{Member} (Membro del Gruppo): può inserire nuove spese specificando chi ha pagato e chi ne ha beneficiato, visualizzare la cronologia delle spese e i saldi del gruppo.
    \item \textbf{Admin} (Amministratore di Gruppo): può approvare nuovi membri, modificare/cancellare qualsiasi spesa del gruppo e gestire le impostazioni del gruppo.
\end{itemize}
I saldi sono per gran parte gestiti dal sistema: esso calcola in in tempo reale il saldo netto di ogni utente e offre una rappresentazione chiara di chi deve e di chi vanta crediti all’interno del gruppo. Il calcolo avviene ottimizzando i rimborsi, in modo da minimizzare il numero di transazioni necessarie per chiudere tutti i conti.

\newpage
\subsection{Architettura}
Il sistema è stato progettato seguendo un'architettura \textbf{multilayer}, che garantisce un netto disaccoppiamento tra l'interfaccia utente, la logica di business e la persistenza dei dati.

\begin{figure}[ht]
    \centering
    \includegraphics[width=1\textwidth]{img/architettura.png}
    \caption{Panoramica dell'architettura logica a livelli del sistema.}
    \label{fig:arch_overview}
\end{figure}

