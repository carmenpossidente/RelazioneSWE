\section{Analisi dei Requisiti}
Questa sezione descrive i requisiti funzionali del sistema attraverso modelli UML 
e artefatti di supporto alla progettazione. Tutti i diagrammi presenti in questa 
sezione e nella successiva sono stati realizzati con \textbf{Lucidchart}, mentre 
i mockup sono stati realizzati con il supporto dell'intelligenza artificiale generativa.

\subsection{Use Case Diagram}
I seguenti diagrammi rappresentano i tre attori del sistema (User, Member, Admin) 
e le principali funzionalità offerte dall'applicazione per ciascun ruolo.

\begin{figure}[H]
    \centering
    \makebox[\textwidth][c]{\includegraphics[width=0.6\textwidth]{img/user_ucdiagram.png}}
    \caption{Use Case Diagram - User}
    \label{fig:use_case_diagram_user}
\end{figure}

\begin{figure}[H]
    \centering
    \makebox[\textwidth][c]{\includegraphics[width=0.6\textwidth]{img/member_ucdiagram.png}}
    \caption{Use Case Diagram - Member}
    \label{fig:use_case_diagram_member}
\end{figure}

\begin{figure}[H]
    \centering
    \makebox[\textwidth][c]{\includegraphics[width=0.6\textwidth]{img/admin_ucdiagram.png}}
    \caption{Use Case Diagram - Admin}
    \label{fig:use_case_diagram_admin}
\end{figure}

\newpage

\subsection{Use Case Templates}

Per ciascun caso d'uso identificato nel diagramma precedente,
sono stati definiti i relativi template descrittivi.
Ogni template specifica attore, pre-condizioni, flusso principale
e flussi alternativi, fornendo una descrizione strutturata
del comportamento atteso del sistema.


% --- UC1: Sign Up ---
\begin{table}[htbp]
    \centering
    \footnotesize
    \begin{tabularx}{\textwidth}{|l|X|}
        \hline
        \multicolumn{2}{|c|}{\textbf{UC1 -- Sign Up}} \\
        \hline
        \textbf{Attore} & User \\
        \hline
        \textbf{Livello} & Function \\
        \hline
        \textbf{Pre-condizioni} & L'utente non deve essere già registrato nel sistema. \\
        \hline
        \textbf{Basic Flow} & 
        \begin{enumerate}[leftmargin=*, nosep, topsep=3pt]
            \item L'utente accede alla pagina iniziale di SplitManager.
            \item Clicca su ``Sign up''.
            \item Il sistema apre la pagina ``Create Account''.
            \item L'utente inserisce i dati richiesti:
            \begin{itemize}[nosep, label={--}, topsep=2pt]
                \item Full name
                \item Email address
                \item Password
                \item Confirm password
            \end{itemize}
            \item Il sistema verifica la validità dei dati (formato email, corrispondenza password).
            \item Il sistema registra il nuovo utente nel database.
            \item Il sistema mostra un messaggio di conferma e reindirizza alla pagina di login (\hyperref[test:uc1_signup]{\textit{Test}}).
        \end{enumerate} \\
        \hline
        \textbf{Alternative Flow} & 
        \begin{enumerate}[label=5\alph*., leftmargin=*, nosep, topsep=3pt]
            \item Se l'email è già registrata, il sistema mostra un messaggio di errore ``Account already exists'' (\hyperref[test:uc1_duplicate_email]{\textit{Test}}).
            \item Se i campi obbligatori non sono compilati o le password non coincidono, il sistema richiede la correzione (\hyperref[test:uc1_invalid_input]{\textit{Test}}).
        \end{enumerate} \\
        \hline
    \end{tabularx}
    \caption{UC1 -- Sign Up}
    \label{tab:uc1}
\end{table}

% --- UC2: Login ---
\begin{table}[htbp]
    \centering
    \footnotesize
    \begin{tabularx}{\textwidth}{|l|X|}
        \hline
        \multicolumn{2}{|c|}{\textbf{UC2 -- Login}} \\
        \hline
        \textbf{Attore} & User \\
        \hline
        \textbf{Livello} & Function \\
        \hline
        \textbf{Pre-condizioni} & L'utente deve essere già registrato nel sistema. \\
        \hline
        \textbf{Basic Flow} & 
        \begin{enumerate}[leftmargin=*, nosep, topsep=3pt]
            \item L'utente accede alla pagina ``Login''.
            \item Inserisce le proprie credenziali (email e password).
            \item Clicca su ``Login''.
            \item Il sistema verifica le credenziali nel database.
            \item Se valide, il sistema apre la Home Page utente, dove può creare o unirsi a gruppi.
        \end{enumerate} \\
        \hline
        \textbf{Alternative Flow} & 
        \begin{enumerate}[label=4\alph*., leftmargin=*, nosep, topsep=3pt]
            \item Se l'email non è registrata, il sistema mostra un messaggio di errore (\hyperref[test:uc2_invalid_email]{\textit{Test}}). 
            Se la password non è corretta, mostra ``Invalid credentials'' e permette un nuovo tentativo (\hyperref[test:uc2_wrong_password]{\textit{Test}}).
            \item Se l'utente dimentica la password, può cliccare su ``Forgot password?'' per avviare la procedura di recupero.
        \end{enumerate} \\
        \hline
    \end{tabularx}
    \caption{UC2 -- Login}
    \label{tab:uc2}
\end{table}

% --- UC3: Create new group ---
\begin{table}[htbp]
    \centering
    \footnotesize
    \begin{tabularx}{\textwidth}{|l|X|}
        \hline
        \multicolumn{2}{|c|}{\textbf{UC3 -- Create New Group}} \\
        \hline
        \textbf{Attore} & User \\
        \hline
        \textbf{Livello} & User Goal \\
        \hline
        \textbf{Pre-condizioni} & L'utente deve essere autenticato (logged in). \\
        \hline
        \textbf{Basic Flow} & 
        \begin{enumerate}[leftmargin=*, nosep, topsep=3pt]
            \item L'utente accede alla dashboard personale (\hyperref[fig:dashboard]{\textit{vedi Mockup \#1}}).
            \item Clicca su ``Create New Group''.
            \item Il sistema apre la pagina di configurazione gruppo.
            \item L'utente inserisce i dettagli del gruppo:
            \begin{itemize}[nosep, label={--}, topsep=2pt]
                \item Group name
                \item Description (facoltativa)
                \item Currency (selezionabile da dropdown)
            \end{itemize}
            \item Clicca su ``Create''.
            \item Il sistema registra il nuovo gruppo, imposta l'utente come Admin, e apre la pagina del gruppo creato (\hyperref[test:uc3_create]{\textit{Test}}).
        \end{enumerate} \\
        \hline
        \textbf{Alternative Flow} & 
        \begin{enumerate}[label=5\alph*., leftmargin=*, nosep, topsep=3pt]
            \item Se mancano dati obbligatori (es.\ nome gruppo), il sistema mostra un messaggio di errore ``Please complete all required fields''.
            \item Se si verifica un errore di rete o salvataggio, viene mostrato ``Group creation failed. Try again later.''
        \end{enumerate} \\
        \hline
    \end{tabularx}
    \caption{UC3 -- Create New Group}
    \label{tab:uc3}
\end{table}

% --- UC4: Join group by invitation ---
\begin{table}[htbp]
    \centering
    \footnotesize
    \begin{tabularx}{\textwidth}{|l|X|}
        \hline
        \multicolumn{2}{|c|}{\textbf{UC4 -- Join Group by Invitation}} \\
        \hline
        \textbf{Attore} & User \\
        \hline
        \textbf{Livello} & User Goal \\
        \hline
        \textbf{Pre-condizioni} & L'utente deve avere un account valido ed essere loggato. Deve aver ricevuto un link o un codice invito generato da un Admin. \\
        \hline
        \textbf{Basic Flow} & 
        \begin{enumerate}[leftmargin=*, nosep, topsep=3pt]
            \item L'utente accede alla sezione ``Join Group'' dalla Dashboard (\hyperref[fig:dashboard]{\textit{vedi Mockup \#1}}).
            \item Inserisce il codice di invito oppure clicca direttamente sul link ricevuto.
            \item Il sistema verifica la validità del codice e l'esistenza del gruppo.
            \item Se valido, il sistema aggiunge l'utente come Member del gruppo in stato \texttt{WAITING\_ACCEPTANCE} (\hyperref[test:uc4_join]{\textit{Test}}).
            \item Il sistema mostra un messaggio di conferma e reindirizza alla pagina del gruppo (\hyperref[fig:group_details]{\textit{vedi Mockup \#2}}).
        \end{enumerate} \\
        \hline
        \textbf{Alternative Flow} & 
        \begin{enumerate}[label=3\alph*., leftmargin=*, nosep, topsep=3pt]
            \item Se il codice è errato o scaduto, il sistema mostra ``Invalid or expired invitation'' (\hyperref[test:uc4_invalid_code]{\textit{Test}}).
            \item Se l'utente è già membro del gruppo, il sistema mostra ``You are already part of this group''.
        \end{enumerate} \\
        \hline
    \end{tabularx}
    \caption{UC4 -- Join Group by Invitation}
    \label{tab:uc4}
\end{table}

% --- UC5: Add new expense ---
\begin{table}[htbp]
    \centering
    \footnotesize
    \begin{tabularx}{\textwidth}{|l|X|}
        \hline
        \multicolumn{2}{|c|}{\textbf{UC5 -- Add New Expense}} \\
        \hline
        \textbf{Attore} & Member \\
        \hline
        \textbf{Livello} & User Goal \\
        \hline
        \textbf{Pre-condizioni} & Il membro deve appartenere ad almeno un gruppo. \\
        \hline
        \textbf{Basic Flow} & 
        \begin{enumerate}[leftmargin=*, nosep, topsep=3pt]
            \item Il membro accede alla pagina del gruppo.
            \item Clicca sul pulsante ``Add Expense'' in alto a destra.
            \item Il sistema apre la finestra modale di inserimento (\hyperref[fig:add_expense]{\textit{vedi Mockup \#3}}).
            \item Il membro inserisce i dettagli della spesa:
            \begin{itemize}[nosep, label={--}, topsep=2pt]
                \item Description 
                \item Amount
                \item Category (tramite dropdown)
                \item Paid by (tramite dropdown)
                \item Split between (checkbox membri o ``Select all'')
            \end{itemize}
            \item Il sistema verifica la correttezza dei dati inseriti.
            \item Il sistema registra la nuova spesa, aggiorna i saldi automaticamente tramite il Pattern Observer, e chiude il modale (\hyperref[test:uc5_balances]{\textit{Test}}).
        \end{enumerate} \\
        \hline
        \textbf{Alternative Flow} & 
        \begin{enumerate}[label=4\alph*., leftmargin=*, nosep, topsep=3pt]
            \item Se l'importo non è valido (negativo o zero), il sistema mostra un messaggio di errore (\hyperref[test:uc5_invalid_amount]{\textit{Test}}).
        \end{enumerate}
        \begin{enumerate}[label=5\alph*., leftmargin=*, nosep, topsep=3pt]
            \item Se i dati sono incompleti, il sistema richiede il completamento dei campi mancanti.
        \end{enumerate} \\
        \hline
    \end{tabularx}
    \caption{UC5 -- Add New Expense}
    \label{tab:uc5}
\end{table}

% --- UC6: View group balances ---
\begin{table}[htbp]
    \centering
    \footnotesize
    \begin{tabularx}{\textwidth}{|l|X|}
        \hline
        \multicolumn{2}{|c|}{\textbf{UC6 -- View Group Balances}} \\
        \hline
        \textbf{Attore} & Member \\
        \hline
        \textbf{Livello} & User Goal \\
        \hline
        \textbf{Pre-condizioni} & Il membro deve appartenere ad almeno un gruppo. \\
        \hline
        \textbf{Basic Flow} & 
        \begin{enumerate}[leftmargin=*, nosep, topsep=3pt]
            \item Il membro accede alla pagina del gruppo.
            \item Seleziona la sezione ``Balances'' (\hyperref[fig:balances]{\textit{vedi Mockup \#4}}).
            \item Il sistema mostra:
            \begin{itemize}[nosep, label={--}, topsep=2pt]
                \item Totale delle spese del gruppo
                \item Saldo individuale di ciascun membro (\hyperref[test:uc6_balances]{\textit{Test}})
                \item Lista di debiti/crediti reciproci ottimizzata da \texttt{MinTransactionsStrategy}
            \end{itemize}
            \item Il membro può visualizzare i dettagli delle transazioni passate.
        \end{enumerate} \\
        \hline
        \textbf{Alternative Flow} & 
        \begin{enumerate}[label=3\alph*., leftmargin=*, nosep, topsep=3pt]
            \item Se non ci sono spese registrate, il sistema mostra ``No expenses yet'' (\hyperref[test:uc6_settled]{\textit{Test}}).
        \end{enumerate} \\
        \hline
    \end{tabularx}
    \caption{UC6 -- View Group Balances}
    \label{tab:uc6}
\end{table}

% --- UC7: View expense history ---
\begin{table}[htbp]
    \centering
    \footnotesize
    \begin{tabularx}{\textwidth}{|l|X|}
        \hline
        \multicolumn{2}{|c|}{\textbf{UC7 -- View Expense History}} \\
        \hline
        \textbf{Attore} & Member \\
        \hline
        \textbf{Livello} & User Goal \\
        \hline
        \textbf{Pre-condizioni} & Il membro appartiene ad un gruppo con almeno una spesa registrata. \\
        \hline
        \textbf{Basic Flow} & 
        \begin{enumerate}[leftmargin=*, nosep, topsep=3pt]
            \item Il membro accede alla sezione ``Expense History'' del gruppo.
            \item Il sistema mostra una lista cronologica delle spese con:
            \begin{itemize}[nosep, label={--}, topsep=2pt]
                \item Data
                \item Descrizione
                \item Importo
                \item Chi ha pagato
                \item Tra chi è stata divisa
            \end{itemize}
            \item Il membro può filtrare o cercare spese specifiche per categoria, data o membro.
        \end{enumerate} \\
        \hline
        \textbf{Alternative Flow} & 
        \begin{enumerate}[label=2\alph*., leftmargin=*, nosep, topsep=3pt]
            \item Se non ci sono spese nel periodo selezionato, il sistema mostra ``No expenses found''.
        \end{enumerate} \\
        \hline
    \end{tabularx}
    \caption{UC7 -- View Expense History}
    \label{tab:uc7}
\end{table}

% --- UC8: Settle debt with a member ---
\begin{table}[htbp]
    \centering
    \footnotesize
    \begin{tabularx}{\textwidth}{|l|X|}
        \hline
        \multicolumn{2}{|c|}{\textbf{UC8 -- Settle Debt with a Member}} \\
        \hline
        \textbf{Attore} & Member \\
        \hline
        \textbf{Livello} & User Goal \\
        \hline
        \textbf{Pre-condizioni} & Il membro ha un debito aperto verso un altro membro del gruppo. \\
        \hline
        \textbf{Basic Flow} & 
        \begin{enumerate}[leftmargin=*, nosep, topsep=3pt]
            \item Il membro accede alla sezione ``Balances'' (\hyperref[fig:balances]{\textit{vedi Mockup \#4}}).
            \item Seleziona il debito da saldare.
            \item Il sistema mostra l'importo dovuto e chiede conferma del pagamento.
            \item Il membro conferma.
            \item Il sistema registra il pagamento in stato \texttt{PENDING} e aggiorna i saldi solo alla conferma definitiva del creditore.
        \end{enumerate} \\
        \hline
        \textbf{Alternative Flow} & 
        \begin{enumerate}[label=4\alph*., leftmargin=*, nosep, topsep=3pt]
            \item Se il membro annulla l'operazione, il sistema chiude la richiesta senza registrare il pagamento e i saldi rimangono invariati (\hyperref[test:uc8_cancel]{\textit{Test}}).
            \item Se l'importo inserito supera il debito reale, il sistema blocca l'operazione (\hyperref[test:uc8_exceeds_debt]{\textit{Test}}).
        \end{enumerate} \\
        \hline
    \end{tabularx}
    \caption{UC8 -- Settle Debt with a Member}
    \label{tab:uc8}
\end{table}

% --- UC9: Invite a new member ---
\begin{table}[htbp]
    \centering
    \footnotesize
    \begin{tabularx}{\textwidth}{|l|X|}
        \hline
        \multicolumn{2}{|c|}{\textbf{UC9 -- Invite a New Member}} \\
        \hline
        \textbf{Attore} & Admin \\
        \hline
        \textbf{Livello} & User Goal \\
        \hline
        \textbf{Pre-condizioni} & L'Admin è autenticato e si trova nella pagina del gruppo. \\
        \hline
        \textbf{Basic Flow} & 
        \begin{enumerate}[leftmargin=*, nosep, topsep=3pt]
            \item L'Admin accede alla sezione ``Members Page'' del gruppo.
            \item Seleziona l'opzione ``Invite a new member''.
            \item Il sistema genera un codice/link di invito univoco (\hyperref[test:uc9_invite]{\textit{Test}}).
            \item L'Admin copia o invia il link ai potenziali membri tramite canali esterni.
            \item Il sistema registra l'invito in stato ``Waiting for acceptance''.
            \item Quando l'utente destinatario utilizza il link, il sistema valida il codice e invia la richiesta di ingresso al gruppo.
            \item L'Admin riceve la notifica di nuova richiesta e potrà successivamente approvarla (vedi UC10).
        \end{enumerate} \\
        \hline
        \textbf{Alternative Flow} & 
        \begin{enumerate}[label=2\alph*., leftmargin=*, nosep, topsep=3pt]
            \item Se esiste già un invito attivo per lo stesso utente, il sistema non genera un nuovo codice e mostra il messaggio ``The invitation has already been sent''.
        \end{enumerate}
        \begin{enumerate}[label=4\alph*., leftmargin=*, nosep, topsep=3pt]
            \item Se il link scade o viene revocato dall'Admin, il sistema mostra ``The invitation has expired''.
        \end{enumerate}
        \begin{enumerate}[label=5\alph*., leftmargin=*, nosep, topsep=3pt]
            \item Se un membro non-admin tenta di generare un codice invito, il sistema rifiuta l'operazione (\hyperref[test:uc9_nonadmin]{\textit{Test}}).
        \end{enumerate} \\
        \hline
    \end{tabularx}
    \caption{UC9 -- Invite a New Member}
    \label{tab:uc9}
\end{table}

% --- UC10: Manage existing members ---
\begin{table}[htbp]
    \centering
    \footnotesize
    \begin{tabularx}{\textwidth}{|l|X|}
        \hline
        \multicolumn{2}{|c|}{\textbf{UC10 -- Manage Existing Members}} \\
        \hline
        \textbf{Attore} & Member (creatore della spesa) o Admin \\
        \hline
        \textbf{Livello} & User Goal \\
        \hline
        \textbf{Pre-condizioni} & L'Admin è autenticato. Esiste almeno un gruppo attivo con uno o più membri. \\
        \hline
        \textbf{Basic Flow} & 
        \begin{enumerate}[leftmargin=*, nosep, topsep=3pt]
            \item L'Admin apre la sezione ``Members Page''.
            \item Il sistema mostra l'elenco dei membri con stato (``Active''/``Waiting'') e saldo corrente.
            \item L'Admin può:
            \begin{itemize}[nosep, label={--}, topsep=2pt]
                \item Approvare/rifiutare membri in attesa
                \item Rimuovere un membro esistente
                \item Modificare i permessi
            \end{itemize}
            \item Il sistema aggiorna automaticamente lo stato del gruppo, i saldi e la cronologia.
            \item L'Admin riceve conferma delle modifiche effettuate.
        \end{enumerate} \\
        \hline
        \textbf{Alternative Flow} & 
        \begin{enumerate}[label=3\alph*., leftmargin=*, nosep, topsep=3pt]
            \item Se l'Admin tenta di rimuovere un membro con debiti aperti, il sistema blocca l'operazione e mostra il messaggio ``The member cannot be removed'' (\hyperref[test:uc10_remove_debt]{\textit{Test}}).
            \item L'approvazione non va a buon fine per link scaduto: il sistema notifica errore ``The invitation is no longer valid''.
        \end{enumerate} \\
        \hline
    \end{tabularx}
    \caption{UC10 -- Manage Existing Members}
    \label{tab:uc10}
\end{table}

% --- UC11: Edit/Delete expense ---
\begin{table}[htbp]
    \centering
    \footnotesize
    \begin{tabularx}{\textwidth}{|l|X|}
        \hline
        \multicolumn{2}{|c|}{\textbf{UC11 -- Edit/Delete Expense}} \\
        \hline
        \textbf{Attore} & Admin \\
        \hline
        \textbf{Livello} & User Goal \\
        \hline
        \textbf{Pre-condizioni} & Il membro è autenticato ed è il creatore della spesa, oppure è l'Admin del gruppo. \\ \\
        \hline
        \textbf{Basic Flow} & 
        \begin{enumerate}[leftmargin=*, nosep, topsep=3pt]
            \item L'Admin accede alla sezione ``Expense page''.
            \item Seleziona la spesa da modificare o eliminare.
            \item Il sistema valida la coerenza dei dati modificati.
            \item Se confermata, il sistema aggiorna i saldi del gruppo tramite il Pattern Observer (\hyperref[test:uc11_edit]{\textit{Test}}) e, in caso di eliminazione, li annulla completamente (\hyperref[test:uc11_delete]{\textit{Test}}).
            \item Il sistema notifica tutti i membri del gruppo.
        \end{enumerate} \\
        \hline
        \textbf{Alternative Flow} & 
        \begin{enumerate}[label=2\alph*., leftmargin=*, nosep, topsep=3pt]
            \item Tentativo di modifica o eliminazione da parte di un membro non creatore della spesa: il sistema blocca l'operazione (\hyperref[test:uc11_unauthorized]{\textit{Test}}).
            \item L'Admin annulla l'operazione: nessuna modifica viene salvata.
        \end{enumerate} \\
        \hline
    \end{tabularx}
    \caption{UC11 -- Edit/Delete Expense}
    \label{tab:uc11}
\end{table}

% --- UC12: Configure group settings ---
\begin{table}[htbp]
    \centering
    \footnotesize
    \begin{tabularx}{\textwidth}{|l|X|}
        \hline
        \multicolumn{2}{|c|}{\textbf{UC12 -- Configure Group Settings}} \\
        \hline
        \textbf{Attore} & Admin \\
        \hline
        \textbf{Livello} & User Goal \\
        \hline
        \textbf{Pre-condizioni} & L'Admin è autenticato e ha creato il gruppo. \\
        \hline
        \textbf{Basic Flow} & 
        \begin{enumerate}[leftmargin=*, nosep, topsep=3pt]
            \item L'Admin apre la sezione ``Group settings''.
            \item Visualizza le attuali impostazioni (nome, descrizione, valuta, regole).
            \item Modifica uno o più campi:
            \begin{itemize}[nosep, label={--}, topsep=2pt]
                \item Group name and description
                \item Currency
                \item Expense splitting rules
                \item Spending limits
                \item Notification settings
            \end{itemize}
            \item Il sistema valida i dati inseriti.
            \item L'Admin conferma le modifiche.
            \item Il sistema salva le nuove impostazioni e le applica ai futuri calcoli di saldo.
        \end{enumerate} \\
        \hline
        \textbf{Alternative Flow} & 
        \begin{enumerate}[label=3\alph*., leftmargin=*, nosep, topsep=3pt]
            \item Errore di validazione (es.\ valuta non supportata o nome vuoto).
            \item Se un membro non-admin tenta di modificare le impostazioni, il sistema rifiuta l'operazione (\hyperref[test:uc12_unauthorized]{\textit{Test}}).
            \item L'Admin annulla l'operazione: nessuna modifica viene applicata.
        \end{enumerate} \\
        \hline
    \end{tabularx}
    \caption{UC12 -- Configure Group Settings}
    \label{tab:uc12}
\end{table}

\clearpage

\subsection{Mockups}
I mockup illustrano una possibile rappresentazione grafica 
dell’interfaccia utente, coerente con i requisiti funzionali definiti. 


% MOCKUP #1: DASHBOARD
\begin{figure}[htbp]
    \centering
    \makebox[\textwidth][c]{\includegraphics[width=1\textwidth]{img/mockup_dashboard.png}}
    \caption{Mockup \#1 -- User Dashboard}
    \label{fig:dashboard}
\end{figure}

\vspace{0.5cm}

% MOCKUP #2: GROUP DETAILS
\begin{figure}[htbp]
    \centering
    \makebox[\textwidth][c]{\includegraphics[width=1\textwidth]{img/mockup_group_page.png}}
    \caption{Mockup \#2 -- Group Details Page}
    \label{fig:group_details}
\end{figure}

\vspace{0.5cm}

% MOCKUP #3: ADD EXPENSE
\begin{figure}[htbp]
    \centering
    \includegraphics[width=0.5\textwidth]{img/mockup_add_expense.png}
    \caption{Mockup \#3 -- Add Expense Modal}
    \label{fig:add_expense}
\end{figure}

\vspace{0.5cm}

% MOCKUP #4: BALANCES
\begin{figure}[htbp]
    \centering
    \makebox[\textwidth][c]{\includegraphics[width=1\textwidth]{img/mockup_balances.png}}
    \caption{Mockup \#4 -- Balances \& Settlement}
    \label{fig:balances}
\end{figure}

\subsection{Page Navigation Diagram}
Il diagramma di navigazione descrive il flusso tra le diverse
pagine dell’applicazione, evidenziando le transizioni possibili
in base alle azioni dell’utente. 

Come evidenziato nel diagramma (Figura \ref{fig:pagenavigationdiagram}), alcune pagine e azioni, 
come la gestione delle impostazioni di gruppo e la gestione dei membri o inviti, 
sono riservate \textbf{solo} agli amministratori del gruppo. 
La modifica o eliminazione delle spese, invece, è permessa sia agli amministratori che ai membri che hanno originariamente creato la spesa.
Questa restrizione visiva e funzionale garantisce che le operazioni critiche o potenzialmente distruttive non siano esposte ai membri standard, 
migliorando la sicurezza e prevenendo azioni accidentali.

\begin{figure}[htbp]
    \centering
    \makebox[\textwidth][c]{\includegraphics[width=0.9\textwidth]{img/page_navigation_diagram.png}}
    \caption{Page Navigation Diagram}
    \label{fig:pagenavigationdiagram}
\end{figure}
